% sections/04_interface.tex

\section{Program Interface: Inputs, Outputs, and Parameters}
\label{sec:interface}

Understanding the interfaces is crucial for monitoring system performance and designing the RL state space.

\subsection{Inputs and Outputs}
The system interacts via standard ROS 2 topics:

\begin{itemize}
    \item \textbf{Inputs (Sensors):}
    \begin{itemize}
        \item \topic{/scan} (\texttt{sensor\_msgs/LaserScan}): 2D LiDAR ranges from Gazebo.
        \item \topic{/tf} \& \topic{/odom}: Coordinate transforms and odometry from the physics engine.
    \end{itemize}
    
    \item \textbf{Outputs (Actuation \& Data):}
    \begin{itemize}
        \item \topic{/cmd\_vel} (\texttt{geometry\_msgs/Twist}): Velocity commands generated by the explorer or RL agent.
        \item \topic{/map} (\texttt{nav\_msgs/OccupancyGrid}): The generated map, used as visual feedback for performance.
    \end{itemize}
\end{itemize}

\subsection{Critical SLAM Parameters}
The following parameters in \node{slam\_toolbox} significantly impact CPU consumption. In our experiments, these are kept constant to isolate the effect of external CPU contention.

\begin{table}[H]
\centering
\begin{tabular}{p{4.5cm} p{2.5cm} p{7.0cm}}
\toprule
\textbf{Parameter} & \textbf{Value} & \textbf{Impact on Resources} \\
\midrule
\param{map\_update\_interval} & 1.0 s & Lower values require frequent grid regeneration, increasing CPU spikes. \\
\param{resolution} & 0.05 m & Finer resolution ($<0.05$) increases memory usage and path planning costs. \\
\param{transform\_publish\_period} & 0.02 s & High frequency ensures smooth TF tree but consumes bandwidth. \\
\param{max\_laser\_range} & 3.5 m & Limits the ray-tracing computation area. \\
\bottomrule
\end{tabular}
\caption{Key configuration parameters affecting computational load.}
\label{tab:slam_params}
\end{table}