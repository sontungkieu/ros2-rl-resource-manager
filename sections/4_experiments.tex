\section{Experiments and Results}
The testing process was conducted using a Multi-process model across three separate Terminal windows.

\subsection{Terminal 1: Launching the Simulation Environment}
This window is responsible for running the Gazebo backend and loading the simulation world.

\begin{bashcode}[title={Terminal 1: Launch Gazebo}]
# Launch Gazebo with TurtleBot3 world
source /opt/ros/jazzy/setup.bash
export TURTLEBOT3_MODEL=burger
ros2 launch turtlebot3_gazebo turtlebot3_world.launch.py
\end{bashcode}

\textbf{Result:} The Gazebo environment displayed successfully. The TurtleBot3 Burger robot appeared at the origin coordinates along with obstacle objects.

\begin{figure}[H]
    \centering
    \includegraphics[width=0.85\linewidth]{images/gazebo.png}
    \caption{Gazebo Harmonic simulation interface with TurtleBot3}
    \label{fig:gazebo}
\end{figure}

\subsection{Terminal 2: Robot Teleoperation}
This window runs the control node, sending velocity signals to the \texttt{/cmd\_vel} topic.

\begin{bashcode}[title={Terminal 2: Teleop Keyboard}]
# Run teleop node
source /opt/ros/jazzy/setup.bash
export TURTLEBOT3_MODEL=burger
ros2 run turtlebot3_teleop teleop_keyboard
\end{bashcode}

\textit{Control Operation:} Using keys \textbf{w, a, s, d, x} to move the robot. The robot responded well to control commands, moving smoothly within the simulated environment.

\subsection{Terminal 3: Data Visualization}
This window runs RViz2 to display LiDAR sensor data.

\begin{bashcode}[title={Terminal 3: RViz2}]
# Run Visualization tool
ros2 run rviz2 rviz2
\end{bashcode}

\textbf{RViz2 Configuration Performed:}
\begin{itemize}
    \item \textbf{Fixed Frame:} Switched to \texttt{odom}.
    \item \textbf{Add Topic:} Added \texttt{/scan} to display LaserScan (environmental scan data).
    \item \textbf{Add RobotModel:} Added the robot model from the \texttt{/robot\_description} topic.
\end{itemize}

\begin{figure}[H]
    \centering
    \includegraphics[width=0.85\linewidth]{images/rviz.png}
    \caption{RViz2 interface displaying Laser scan data (red points)}
    \label{fig:rviz}
\end{figure}